\documentclass[]{IEEEtran}

\title{Ottimizzazione distribuita di sintesi hardware tramite approccio map-reduce con framework Hadoop}
\author{Andrea Caucchiolo - VR436780\\Elia Brentarolli - VR433534}

\usepackage{graphicx}
\usepackage[english,italian]{babel}
\usepackage[utf8]{inputenc}
\usepackage{caption}
\usepackage{amsmath}
%\usepackage{hyperref}
\usepackage{uri}
\usepackage{xcolor}
%\usepackage{textcomp}
\usepackage{listings}
\definecolor{dkgreen}{rgb}{0,0.6,0}
\definecolor{gray}{rgb}{0.5,0.5,0.5}
\definecolor{mauve}{rgb}{0.58,0,0.82}
\lstdefinestyle{javaStyle}{
	language=Java,
	aboveskip=3mm,
	belowskip=3mm,
	showstringspaces=false,
	columns=flexible,
	basicstyle={\small\ttfamily},
	numbers=none,
	numberstyle=\color{black},
	keywordstyle=\color{blue},
	commentstyle=\color{dkgreen},
	stringstyle=\color{mauve},
	breaklines=true,
	breakatwhitespace=true,
	tabsize=3,
	escapechar=^
}
\begin{document}
\maketitle

\begin{abstract}
	
Questo documento contiene una relazione su come è stato implementato un programma per ottimizzare la sintesi hardware partendo dal \emph{control flow graph} di un software tramite un  approccio di tipo map-reduce e sfruttando al massimo le capacità di calcolo distribuito offerte dal framework. In particolare il programma proposto, preso in input un \emph{control flow graph} del software da sintetizzare e il livello di parallelismo voluto, detto $n$, produce in output $n$ grafi temporizzati scelti tra quelli che rendono minima la funzione di costo.

\end{abstract}


\section{Introduzione}

La possibilità di generare in modo automatico la descrizione di una piattaforma hardware che esegue le stesse operazioni di un software partendo dallo stesso è, ad oggi, una tecnica ancora troppo imprecisa e inaffidabile per poter essere impiegata su larga scala. In particolare per poter approcciarsi a questa tecnica è necessario, tra le varie problematiche, affrontare un problema di ricerca della migliore combinazione dei nodi di un grafo tra tutte le varie permutazioni possibili dello stesso. Ovviamente la risoluzione di un problema del genere per grafi con milioni di permutazioni richiede delle risorse di tempo troppo elevate perché questo possa essere fatto su una singola macchina. Proprio per questo motivo si è deciso di proporre un'implementazione dell'algoritmo di risoluzione basato su cluster e quindi andare a potenziare le risorse disponibili andando ad eseguire le computazioni su un'insieme di macchine in modo da poter ottenere risultati di istanze del problema troppo complesse per macchine singole in tempi non diversi ad quelli delle stesse. Inoltre l'approccio distribuito, a differenza della singola macchina, è molto più scalabile e quindi potenziabile per ottenere risultati sempre migliori a problemi sempre più complessi andando a mantenere tuttavia il costo per l'implementazione relativamente basso.

\section{Conoscenze preliminari e terminologia usata}

Per avere una migliore comprensione della relazione di seguito sono riportati alcuni concetti chiave le relative spiegazioni:
\begin{itemize}
	\item High Level Synthesis (HLS) \cite{HLS}: La sintesi ad alto livello è un'insieme di procedure che permettono di passare da un software scritto in un linguaggio ad alto livello alla descrizione di una piattaforma hardware in grado di eseguire le stesse operazioni del software di partenza e rispettando al contempo i vincoli imposti sulla velocità delle operazioni, le dimensioni del chip e il costo per la produzione. La figura \ref{hls} mostra i passaggi chiave della procedura. Nel corso della relazione si farà riferimento solamente alla fase di scheduling in quanto rappresenta la parte più difficile da affrontare.
	\begin{figure}[htp]
		\includegraphics[width=0.5\textwidth]{images/Hls_flow.png}
		\caption{Schema del processo di \emph{high level synthesis}}
		\label{hls}
	\end{figure}
	
	\item Data Flow Graph: In questa relazione un data flow graph rappresenta un grafo ottenuto partendo dall'analisi di un software generico e che contiene le informazioni sulle dipendenze delle operazioni e sulle componenti che devono essere usate. Qui un Data Flow Graph è definito formalmente come:
	\it{DFG} \normalfont $=(V,E)$ dove $\forall v_i\in V : v_i = (n,o)$ con $n$ rappresentante il nome del nodo, \emph{univoco} all'interno del grafo, e $o$ rappresentante invece il tipo di operazione che il nodo esegue. Infine si ha che $E = {(v_i,v_j)}, v_i,v_j \in V$ tali che $(v_i,v_j) \in E \iff v_j$ dipende da $v_i$ ovvero $v_j$ usa un risultato prodotto dalla computazione di $v_i$. Si vuole infine sottolineare che per come è stato definito un DFG non è possibile avere cicli nel grafo poiché ciò implicherebbe una dipendenza ciclica tra le istruzioni del software di partenza, cosa che implicherebbe l'incorrettezza dello stesso.
	
	\item Grafo temporizzato: Durante la relazione si farà più volte accenno al concetto di grafo temporizzato ottenuto dalla elaborazione di un DFG. Un grafo temporizzato non è altro infatti che un DFG arricchito con delle informazioni sulla tempistica delle operazioni, ovvero il tempo minimo e massimo ai quali l'operazione contenuta nel nodo può essere fatta senza che si sovrapponga a quelle dei propri padri o figli. Formalmente si ha:
	\it{TG} \normalfont $=(V,E)$ con $\forall v_i\in V : v_i = (n,o,t,T,m)$, dove $n,o$ hanno lo stesso significato che hanno nella definizione del DFG, $t$ rappresenta il tempo \emph{minimo} in cui l'operazione può essere fatta, $T$ rappresenta invece il tempo \emph{massimo} in cui l'operazione può essere svolta e $m$ rappresenta la mobilità del nodo nel tempo rispetto a $t$, ovviamente mantenendo sempre $m+t\le T$. Vale inoltre che $T - t = m$. L'insieme $E$ è definito invece in modo del tutto analogo a come fatto con DFG ma con la conseguenza che deve valere $\forall (v_i,v_j) \in E : t(v_i)<t(v_j) \land T(v_i)<T(v_j)$.
	Infine si dice che un grafo temporizzato è una \emph{permutazione} di un'altro grafo se per ogni nodo del primo grafo esiste un corrispettivo nodo nel secondo che differisce esclusivamente per il valore di $m$
	
	\item Map-Reduce\cite{MAPRED}: Paradigma di programmazione per il calcolo distribuito, consiste nel dividere un input di grandi dimensioni i diverse porzioni che vengono poi processate tutte in modo indipendente dalle altre (map). In seguito i risultati parziali vengono raccolti, ordinati e redistribuiti tra le macchine in base al valore impostato come chiave (shuffle and sort) per poi essere combinati in un unico risultato finale. La figura \ref{mapred} schematizza il processo appena descritto.
	\begin{figure}[htp]
		\includegraphics[width=0.5\textwidth]{images/mapred.png}
		\caption{Schema del processo di map-reduce. L'input è diviso tra i vari mapper che producono una serie di risultati del tipo chiave-valore. Successivamente i risultati sono raccolti e ridivisi tra i reducer in base al valore della loro chiave.}
		\label{mapred}
	\end{figure}
	
	\item Hadoop \cite{HADOOP}: Framework di Apache per Java per il calcolo distribuito e che si presta molto bene per l'approccio di tipo map-reduce. Sebbene il framework sia costituito da svariate componenti quali ad esempio l'Hadoop Distributed File System (HDFS) e molto altro, per la comprensione di questa relazione è necessario conoscere solo le classi principali riguardanti l'approccio map-reduce e che verranno spiegate nel dettaglio nel seguito della relazione.
\end{itemize}

\section{Definizione del problema affrontato}
Per poter capire meglio il problema che ci si è posto è necessario prima di tutto spigare molto brevemente in cosa consiste il flusso di progettazione di un sistema embedded (si prenda la figura \ref{pse} come riferimento a quanto detto in seguito).

Per prima cosa  dai requisiti forniti viene creata un'applicazione software che descrive il sistema da implementare. Successivamente questa applicazione viene partizionata in due sottoinsiemi: una parte delle funzionalità diventerà il software in funzione sulla piattaforma mentre la parte rimanente verrà sintetizzata in componenti hardware che verranno installate sulla piattaforma. Il modo in cui il partizionamento viene fatto dipende prevalentemente dai requisiti imposti e dal risultato di vari profiling e simulazioni. È importante notare come questa divisione non sia irreversibile ma, al contrario, è spessa soggetta a cambiamenti dovuti proprio ai risultati dei vari test fatti. Una volta che si è giunti ad una situazione ottima si passa all'implementazione fisica della piattaforma.
\begin{figure}[htp]
	\includegraphics[width=0.5\textwidth]{images/embedded_sys_df.png}
	\caption{Flusso di progettazione di un sistema embedded.}
	\label{pse}
\end{figure}

Fatta questa premessa, ci si vuole ora concentrare sulla parte di sintesi hardware che può essere fatta in solo due modi: il primo, previsto dallo stato dell'arte attuale, prevede la traduzione manuale del codice. Ciò garantisce un maggiore controllo sul risultato finale. Il secondo metodo consiste invece di affidarsi a tool automatici per l'\emph{high level synteshis} che risulta essere molto più veloce ma anche meno controllabile nella qualità del risultato. Come già mostrato in figura \ref{hls}, il processo di HLS prevede una serie di fasi per poter ottenere una descrizione hardware a partire da un codice sorgente. Tra questa fasi sicuramente la più importante è quella che riguarda lo \emph{scheduling} (anche se in figura scheduling e resource alloaction condividono la stesso posto nel diagramma, lo stato dell'arte prevede che prima venga fatto lo scheduling e solo in un secondo momento vengano allocate le risorse in base al risultato precedentemente ottenuto, motivo per cui in questa relazione si parlerà esclusivamente di scheduling e mai di resource allocation).

La fase di scheduling consiste fondamentalmente in tre passaggi con tre algoritmi diversi applicati in sequenza:
\begin{itemize}
	\item  Scheduling ASAP: Lo scheduling ASAP è il primo passa per trasformare un DFG in un grafo temporizzato. Con questo algoritmo si impone che tutte le operazioni dei nodi vengano fatte nel primo tempo utile senza infrangere nessun vincolo, andando così a determinare il valore dell'attributo $t$ dei nodi del TG risultante ed ottenendo di conseguenza il tempo impiegato dal circuito.
	\item Scheduling ALAP: Lo scheduling ALAP riceve come input il risultato del precedente scheduling ed esegue l'operazione diametralmente opposta, poiché viene imposto che tutte le operazioni dei nodi vengano eseguite nell'ultimo tempo utile senza infrangere nessun vincolo ed andando così a determinare il valore dell'attributo $T$ di tutti i nodi del grafo risultato.
	\item Fase di ricerca: Una volta ottenuti i valori di tempo minimo e massimo per ogni nodo incomincia una fase di ricerca in cui vengono spostati tutti i nodi con lo scopo di andare a minimizzare una data funzione di costo (in questa relazione la funzione di costo è data dal numero di nodi che lavorano in parallelo, ovvero dal numero di nodi che operano allo stesso tempo).
\end{itemize}
Le figure \ref{scheduling} e \ref{scheduling2} mostrano un esempio dell'applicazione di quanto riportato sopra.
\begin{figure}[htp]
	\includegraphics[width=0.5\textwidth]{images/schedule.png}
	\caption{Trasformazione di un codice sorgente in DFG e applicazione degli algoritmi ASAP e ALAP (le variabili usate nei nodi sono state omesse dopo lo scheduling asap).}
	\label{scheduling}
	\includegraphics[width=0.3\textwidth]{images/scheduling2.png}
	\caption{Possibile risultato dello scheduling applicato al codice della figura \ref{scheduling}}
	\label{scheduling2}
\end{figure}

In questa relazione si proporrà quindi un algoritmo che, sfruttando l'approccio distribuito basato su map-reduce e le funzionalità del framework Hadoop, cercherà di dare una soluzione ottima al problema della ricerca del grafo ottimo.
\section{Approccio al problema}
Definito il problema, si può passare ora a descrivere l'approccio usato per risolverlo.

Il primo passo fatto è stato quello di andare ad inquadrare il problema in un contesto di tipo map-reduce andando quindi a mappare le fasi che l'algoritmo prevede su quelle del paradigma usato, riservandosi anche di effettuare più cicli di map-reduce qualora necessario.
\subsection{Trasposizione in Map-Reduce}
Trasporre il problema in modo da adattarlo al paradigma map-reduce non è un'operazione complicata. Supponendo di ricevere come input già un DFG, le uniche fasi da implementare sono i tre algoritmi di scheduling, due dei quali tuttavia, ovvero ASAP e ALAP, possono essere applicati in sequenza e non richiedono troppe risorne nè di computazione nè di tempo. La parte più onerosa in termini di costo di computazione riguarda infatti la fase di ricerca dove è necessario andare ad esplorare tutti i possibili grafi validi e calcolare il costo di ognuno, ed è quindi questa fase che si vuole andare ad ammortizzare tramite il calcolo distribuito. 

Per ottenere questo risultato si è partiti da una prima idea di soluzione valida, ovvero di distribuire l'onere del calcolo delle varie permutazioni tra le varie macchine del cluster, e la si è andata a rifinire e modificare lungo tutto il corso del lavoro. Di seguito sono riportate le tre fasi principali cui la soluzione è andata incontro.
\begin{itemize}
	\item Fase 1(fig.\ref{sol_fase1}): In questa prima fase la soluzione consiste essenzialmente in tre passaggi distinti. In un primo passaggio il DFG è processato da un'unica unità centrale con lo scopo di andare a calcolare le varie permutazioni dello stesso. Successivamente queste permutazioni sono inviate ai vari mapper che procedono a calcolarne il costo ed a inviarlo ad un unico reducer che cerca la soluzione a costo migliore. Ovviamente il fatto di dover calcolare tutte le permutazioni in un'unica entità centrale non ha garantito buone prestazioni e ciò a fatto sì che la soluzione venisse migliorata nella seconda fase.
	\begin{figure}[htp]
		\includegraphics[width=0.5\textwidth]{images/sol_fase1.png}
		\caption{Schema della soluzione nella prima fase.}
		\label{sol_fase1}
	\end{figure}
	
	\item Fase 2(fig.\ref{sol_fase2}): Questa fase mantiene lo scheletro della precedente ma sposta il calcolo delle permutazioni sui mapper. Adesso l'unità centrale deve solamente stimare il numero di permutazioni che possono essere generate, dividere tale numero per il numero di macchine presenti e dire ai mapper quante permutazioni calcolare e da quale partire (per maggiori dettagli si veda la sezione \ref{Stima}). Di conseguenza in questa fase si ha un potenziamento dei mapper che devono non solo calcolare tutta una serie di permutazioni di un grafo (eliminando eventuali comfigurazioni illegali) ma per ognuna di queste deve anche calcolarne il relativo costo da inviare al reducer che, rispetto alla fase precedente, non ha subito essenziali modifiche. Già con queste modifiche la soluzione si presenta più performante rispetto alla precedente ma, a causa della struttura con cui hadoop è creato, non è implementabile. Si è dovuto quindi effettuare una seconda ristrutturazione per rendere la soluzione compatibile con il framework.
	\begin{figure}[htp]
		\includegraphics[width=0.5\textwidth]{images/sol_fase2.png}
		\caption{Schema della soluzione nella seconda fase.}
		\label{sol_fase2}
	\end{figure}
	
	\item Fase 3(fig.\ref{sol_fase3}): Questa fase, seppur mantenendo i concetti espressi nella seconda, vede uno shift a livello di passaggi coinvolti. Adesso infatti l'unità centralizzata responsabile del calcolo delle permutazioni viene completamente eliminata e le sue funzionalità sono ereditate da un singolo mapper che calcola il punto di partenza e il numero di permutazioni da inviare ai reducer. Questi infatti ereditano le funzionalità dei mapper e devono calcolare la migliore permutazione tra tutte quelle loro assegnate. Ciò comporta che l'output del job non è più un unico grafo come in precedenza ma una numero di grafi pari al numero di reducer ed è quindi necessario un modo per cercare il migliore. Questo può essere fatto tramite un semplice job di map-reduce oppure (in caso di un numero ristretto di output) con un semplice programma sequenziale.
	\begin{figure}[htp]
		\includegraphics[width=0.5\textwidth]{images/sol_schema3.png}
		\caption{Schema della soluzione nella terza fase.}
		\label{sol_fase3}
	\end{figure}
\end{itemize}
\subsection{Implementazione con Hadoop}
Dopo aver definito lo schema della soluzione da adottare si è passati all'effettiva implementazione in Hadoop. Poiché il framework supporta pienamente il paradigma map-reduce è stato sufficiente andare ad implementare le varie classi necessarie perché il job venisse accettato, ovvero le classi java \emph{Driver}, \emph{InputReader},\emph{Mapper},\emph{Partitioner} e \emph{Reducer} (approfondite nella sezione \ref{work}). Se l'implementazione delle classi è stata facile, ci si è dovuti però andare a scontrare con alcune delle limitazioni imposte dal framework, limitazioni che in più punti hanno costretto a modificare la soluzione adottata o a imporre delle restrizioni in modo forzato.

La prima limitazione incontrata è dovuta all'impossibilità di conoscere a runtime il numero di nodi che il cluster possiede. Senza questa informazione non sarebbe potuto suddividere le permutazioni tra le macchine e, di conseguenza, non sarebbe stato possibile risolvere il problema nel modo desiderato. Per porre rimedio a ciò si è scelto di imporre all'utente di inserire, al lancio del job, la granularità del parallelismo voluto, ovvero il numero per cui le permutazioni stimate vanno divise e quindi la quantità di lavoro assegnata ad ogni singolo nodo. Sebbene in un primo momento questa scelta possa sembrare una limitazione eccessiva, una volta implementata la terza soluzione, ci si è resi conto che venivano introdotti come effetto collaterale due importanti vantaggi. Con questo metodo è possibile, in primo luogo, scegliere di non usare tutta la potenza di calcolo del cluster ma solo una frazione, lasciando liberi gli altri nodi di eseguire altri job. In seconda istanza si può, al contrario di prima, andare ad inserire un valore che supera il numero di nodi nel cluster. In questo modo ogni nodo si troverà a dover eseguire un numero minore di operazioni e, di conseguenza, i nodi più veloci riusciranno a completare più task di reduce rispetto a quelli lenti che faticano a completarne uno. Grazie a ciò si crea un meccanismo semi-automatico del bilanciamento del carico di lavoro al prezzo di un aumento del numero dei file di output (ogni task di reduce genera un file di output).
\begin{figure}
	\includegraphics[width=0.5\textwidth]{images/exec.png}
	\caption{Comando per lanciare il job su Hadoop. Si noti il numero ``10'' posto alla fine ad indicare di dividere le permutazioni per 10 cluster. Inoltre se possibile verranno usati 10 nodi per eseguire i task di reduce (in caso il numero sia eccessivo Hadoop allocherà quanto disponibile senza ritornare errore).}
\end{figure}

Un'altro ostacolo incontrato durante l'implementazione del software è stato la difficoltà con cui Hadoop permette di scegliere il numero di mapper da usare. Se infatti Hadoop permette in modo molto facile di manipolare il numero di reducer usati, il modo in cui il numero di mapper è calcolato è difficilmente influenzabile dal programmatore. Questo è dovuto al fatto che il framework, per come è stato concepito, si aspetta di processare in input o un'unico file di grandi dimensioni oppure diversi file in quanto, sostanzialmente,il numero di mapper è calcolato comela somma delle dimensioni in byte di tutti i file diviso una data costante. Tuttavia nel nostro caso capita sempre che i file usati come input e contenenti il DFG iniziale siano sempre troppo piccoli rispetto alla costante impostata e di conseguenza si ottiene sempre l'allocazione di un solo mapper. Per risolvere questo problema si sono applicate due diverse soluzioni a seconda della fase della soluzione in cui si stava lavorando. Nella prima e seconda fase ciò che è stato fatto è stato di andare a manipolare la costante usata nella divisione impostandola al rapporto tra la dimensione totale dell'input, letta precedentemente, e il numero di mapper voluto. In questo modo il valore del numero di mapper che Hadoop imposta era esattamente il valore voluto.

Nel momento in cui si è passati alla terza fase della soluzione questo procedimento non era però più necessario in quanto ora bisognava solo accertarsi che il mapper allocato fosse esattamente uno. Dopo un'attenta analisi del codice di Hadoop, in particolare la parte riguardante la classe \emph{FileInputFormat}\cite{FILEINPUTFORMAT} e \emph{JobConf}\cite{JOBCONF} si è potuto notare come il numero di mapper sia calcolato da Hadoop. Molto semplicemente il framework prende ogni file in input, verifica se è possibile fare lo split e, se possibile, lo divide ed inserisce le partizioni all'interno di un'apposita lista, altrimenti il file non viene diviso ma comunque viene aggiunto un elemento all'interno della lista precedentemente citata. La dimensione di tale lista è poi usata da Hadoop per allocare i mapper. Ciò che è stato fatto quindi è stato solamente andare a sovrascrivere il metodo booleano per il controllo dello splitting andando a fare in modo che ritornasse sempre false. In questa maniera, e poiché il software accetta solo un file in input, sia che l'input sia troppo piccolo o troppo grande nella lista degli split verrà inserita una ed una sola entry e di conseguenza verrà allocato solo un mapper.

Ultimo ostacolo affrontato è stato il modo in cui il framework gestisce la suddivisione degli input. Come si è potuto osservare in precedenza, sia la prima che la seconda fase della soluzione proposta si basavano fortemente sull'uso di un nodo centrale per il parsing e la suddivisione dei grafi tra i mapper. Inizialmente si era pensato di affidare il compito alla classe responsabile di leggere il file di input nell'errata convinzione che questa fosse univoca all'interno del framework. Dopo alcuni test e ricerche si è però capito che in realtà questa classe non è unica rispetto al framework ma è univoca rispetto al mapper, ovvero ogni mapper istanzia una classe propria che legge l'input, anch'esso replicato sulle diverse macchine. A causa di ciò il lavoro che doveva essere fatto solo una volta veniva moltiplicato per il numero di mapper allocati, rompendo quindi la correttezza del programma. A causa di ciò si è decisi di modificare il programma, passando dunque alla terza fase, e di delegare il compito della suddivisione delle partizioni al mapper, il quale può essere univoco all'interno del framework.
\begin{figure}
	\includegraphics[width=0.5\textwidth]{images/hadoop_input.png}
	\caption{A sinistra il modo in cui l'input doveva essere processato secondo la soluzione nella sua prima e seconda fase, a destra il modo in cui Hadoop lo gestiva. Si noti che in questo modo ogni mapper esegue tutto il lavoro e non solo una porzione.}
\end{figure}

\section{Lavoro svolto} \label{work}
Per implementare soluzione proposta tramite Hadoop è stato sufficiente andare ad implementare tutte le classi che il framework richiede per poter sottomettere un job  di map-reduce al cluster. Di seguito sono riportate tutte le classi implementate con le relative spiegazioni e porzioni di codice.
\subsection{Classe Driver}
In un job di Hadoop la classe \emph{Driver} ha il compito di andare a preparare  l'ambiente in cui l'applicazione verrà lanciata andando configurare tutte le variabili necessarie. La classe Driver implementata nel software proposto non si distacca da questa molto da questo paradigma ma presenta comunque alcuni punti degni di nota.

Il primo punto importante riguarda i parametri in input. Normalmente un'applicazione Hadoop prevede come input solamente il path per l'input e quello per l'output ma nel nostro caso è stato aggiunto un terzo parametro, ovvero il livello di parallelismo voluto, che deve quindi essere letto e inserito all'interno di una variabile di ambiente in modo che poi possa essere recuperato successivamente nel mapper dato che non è possibile alcun tipo di comunicazione diretta tra la classe driver e le altri classi del framewrok.
\begin{lstlisting}[style=javaStyle]

	if (args.length != 3) {
		System.err.println("Usage: ExplorerDriver <input path> <outputpath> <#_nodes_to_use>");
		System.exit(-1);
	}
	...
	//set number of reducers by casting 3rd agument
	int nodes=Integer.parseInt(args[2]);
	//set the variabile "nodes" in the job configuration class
	conf.setInt("nodes", nodes);
\end{lstlisting}

Secondo punto degno di nota è dato dal fatto che il Driver ha il compito di andare ad indicare al framework tutte le classi utili per lo svolgimento del programma così come il tipo dei dati prodotti in output ed il numero di reducer usati:
\begin{lstlisting}[style=javaStyle]
//number of reducers
jobConf.setNumReduceTasks(nodes); 
...
//job name displayed
job.setJobName("Graph Exploration"); 
...
//Iinput path of the file
FileInputFormat.addInputPath(job, new Path(args[0])); 
//Input parser class
job.setInputFormatClass(GraphInputFormat.class); 
//Output directory
FileOutputFormat.setOutputPath(job, new Path(args[1]));
//Mapper class
job.setMapperClass(ExplorerMapper.class);	
//Reducer class
job.setReducerClass(ExplorerReducer.class); 
//Partitioner class
job.setPartitionerClass(ExplorerPartitioner.class); 
//Key and Value types
job.setOutputKeyClass(Text.class); 
job.setOutputValueClass(Text.class);
\end{lstlisting}
\subsection{Classe GraphInputReader}
La classe GraphInputReader è una classe che estende l'originaria classe RecordReader di Hadoop. Tale classe ha, in origine, lo scopo di leggere porzioni di file da inviare al mapper associatogli (tipicamente riga per riga) fino all'esaurimento del file in input. Tuttavia poiché questo comportamento non si adattava bene a quello del programma è stato scelto di sostituirla con una classe apposita con lo scopo di leggere in un unico colpo tutto il file. TODO INSERIRE COSE SU GRAFI E CATENE

\subsection{Classe Partitioner}
La classe \emph{Partitioner} ha il compito, all'interno del framework, di andare a distribuire i risultati dei mapper tra i vari reducer. Tipicamente questo è fatto andando a creare l'hash delle chiave usando il resto della divisione per il numero di reducer come identificativo del reducer cui il dato deve essere girato. In questo caso però ciò non è necessario poiché il numero di risultati è esattamente uguale al numero di reducer usati, quindi ogni risultato è mappato univocamente su un solo reducer tramite un semplice contatore che va da 0 al numero di questi meno uno.
\begin{lstlisting}[style=javaStyle]
public class ExplorerPartitioner extends Partitioner<Text,Text>{

	private Integer counter;
	@Override
	public int getPartition(Text key, Text val, int partitions) {
		return (counter == null ? counter = 0 : ++counter);
	}
}
\end{lstlisting}

\section{Risultati ottenuti}
%TODO Discutere risultati ottenuti con algoritmi su cluster mettendo in relazione permutazioni/tempo impiegato e nodi usati con screenshot.
\section{Conclusioni}

\bibliographystyle{IEEEtran}
\bibliography{biblio}

\end{document}